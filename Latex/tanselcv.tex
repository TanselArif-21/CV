\documentclass[11pt,a4paper]{moderncv}

\hyphenpenalty 1000

% moderncv themes
%\moderncvtheme[blue,roman]{classic}                  % optional argument are 'blue' (default), 'orange', 'green', 'red', 'purple', 'grey' and 'roman' (for roman fonts, instead of sans serif fonts)
%\moderncvtheme[green]{classic}                % idem
\moderncvstyle{classic}
\moderncvcolor{blue}  
% character encoding
%\usepackage[utf8]{inputenc}                   % replace by the encoding you are using

% adjust the page margins
\usepackage[scale=0.85]{geometry}
\setlength{\hintscolumnwidth}{2.5cm}
%\usepackage{fontawesome}					
% if you want to change the width of the column with the dates
%\AtBeginDocument{\setlength{\maketitlenamewidth}{6cm}}  % only for the classic theme, if you want to change the width of your name placeholder (to leave more space for your address details
\AtBeginDocument{\recomputelengths}                     % required when changes are made to page layout lengths
 
% Hyperlinks
%\usepackage{hyperref}								% to use hyperlinks
\definecolor{moderncv-blue}{rgb}{0.22,0.45,0.70}
\definecolor{moderncv-grey}{rgb}{0.55,0.55,0.55}
\definecolor{moderncv-orange}{rgb}{0.95,0.55,0.15}
\definecolor{moderncv-green}{rgb}{0.35,0.70,0.30}
\definecolor{linkcolour}{rgb}{0,0.2,0.6}			% hyperlinks setup
\AfterPreamble{\hypersetup{colorlinks,breaklinks,urlcolor=moderncv-blue, linkcolor=moderncv-blue}}

% personal data
\firstname{Tansel}
\familyname{Arif}
%\title{Cypriot}               % optional, remove the line if not wanted
\address{37 All Saints Court, Didcot}{Oxfordshire OX11 7NG, U.K.}    % optional, remove the line if not wanted
%\phone{+44 (0)78 3591 1553}                    % optional, remove the line if not wanted
%\mobile{+30 698 4385057}                      % optional, remove the line if not wanted
%\fax{fax (optional)}                          % optional, remove the line if not wanted
\email{tansel.arif@live.co.uk}                      % optional, remove the line if not wanted
\social[github][www.github.com/TanselArif-21]{www.github.com/TanselArif-21}
\homepage{www.linkedin.com/in/tansel-arif-86264492} % optional, remove the line if not wanted
%\photo[64pt][0.4pt]{picture}                         % '64pt' is the height the picture must be resized to, 0.4pt is the thickness of the frame around it (put it to 0pt for no frame) and 'picture' is the name of the picture file; optional, remove the line if not wanted
%\quote{Some quote (optional)}                 % optional, remove the line if not wanted

% to show numerical labels in the bibliography; only useful if you make citations in your resume
\makeatletter
\renewcommand*{\bibliographyitemlabel}{\@biblabel{\arabic{enumiv}}}
\makeatother

% bibliography with mutiple entries
%\usepackage{multibib}
%\newcites{book,misc}{{Books},{Others}}

%\nopagenumbers{}                             % uncomment to suppress automatic page numbering for CVs longer than one page
%----------------------------------------------------------------------------------
%            content
%----------------------------------------------------------------------------------
\begin{document}
\maketitle
\section{Skills}
\cvline{Programming}{SQL/T-SQL, Python, Git}
\cvline{Other}{Pandas, Numpy, Scikit-Learn, Keras}
\cvline{General}{Numerical computing,  Strong ad hoc problem solving}

\section{Experience}
\cventry{Mar 2019 -- Present}{Unilever - Data Science and Analytics Manager}{}{UK}{}{\normalsize
Responsibilities: 
\begin{itemize}
\item[-] Identifying areas of interests to stakeholders, scheduling and leading workshops leveraging the teams skills to deliver key insights with clear goals and outcomes.
\item[-] Developing end-to-end Machine Learning solutions.
\item[-] Ensuring members of the team have the support, guidance and direction they need to accomplish their goals.
\item[-] Supporting the recruitment process from a statistical perspective, ensuring it is bias-free, efficient and cost effective.
\item[-] Training and up-skilling the team in areas such as Data Science, Python, SQL and Git.
\end{itemize}}

\cventry{Jun 2018 -- Aug 2018}{Thought Provoking Consulting - Quantitative Consultant, Data Scientist}{}{UK}{}{\normalsize
Responsibilities: 
\begin{itemize}
\item[-] Inference methods (Bayesian) - R.
\item[-] EDA and machine learning (Linear Regression, NLP) - Python.
\item[-] Implementing optimisation algorithms (algorithms developed to optimise a target indicator) - C\#.
\item[-] Creating and maintaining proper source control, deployment and maintenance of code for in-house tools.
\end{itemize}}

\cventry{Dec 2017 -- May 2018}{FIS (SunGard) - Quantitative Consultant}{}{UK}{}{\normalsize
Responsibilities: 
\begin{itemize}
\item[-] Specification and implementation of mathematical models using C\# for the efficient pricing of complex financial products, for the evolution of future market and credit events and for the calibration of risk models.
\item[-] Verifying that new and existing models are correct and appropriate.
\item[-] Providing client support on questions related to software behaviour.
\item[-] Project management in times of scarce resources.
\end{itemize}}

\cventry{Sep 2015 -- Dec 2017}{FIS (SunGard) - Consultant, Risk and Compliance}{}{UK}{}{\normalsize Previously SunGard Financial Systems. A vendor providing solutions to financial corporations in terms of risk and exposure management and financial regulatory compliance. Responsibilities: 
\begin{itemize}
\item[-] Maintenance, optimisation and troubleshooting of test farms / servers / databases which clients use for test cases for product development using Delphi and T-SQL (Microsoft SQL Server).
\item[-] Finding and carrying out optimisations and fixes to these environments
\item[-] Implementing code changes (Pascal/C\#) to improve or fix issues in calculation methodology/equations
\item[-] Customisation of the user facing web code to suit the needs and requirements of users (Javascript/C\#)
\item[-] Coding and producing independent support utilities to improve client satisfaction
\end{itemize}}

%%% WORK AND RESEARCH
\section{Academia}
\cventry{2011 -- 2015}{Imperial College London}{PhD. Materials Science and Engineering}{UK}{}{\normalsize 
\begin{itemize}
\item[-] The focus during my PhD research has been on the development of theory and code (C++) for the phase-field modelling and simulation of microstructures found in steel~\hyperref[refbain]{[1,2]} as well as the formation of van der Waals fluids using the smoothed particle hydrodynamics method. 
\item[-] Given my interest in the prediction of general evolutionary phenomena, I have collaborated on cellular automata treatment for solidification~\hyperref[refmeas]{[3]}. 
\item[-] My final results involve the development of tools to combine the capabilities of multiple models to deal with situations involving fluid flow, solidification and solid-state phase transformations.
\end{itemize}}
\cventry{2009 - 2010}{Queen Mary University of London}{MSci. ($1^{st}$ Class Hons) Mathematics}{UK}{}{}
%{\normalsize Supervisor: Prof. P. Cameron}
\cventry{2006 - 2009}{Queen Mary University of London}{BSc. ($1^{st}$ Class Hons) Mathematics}{UK}{}{}
{}

\section{Training}
\cvline{February 2020}{\textbf{Structuring Machine Learning Projects}[\href{https://www.coursera.org/account/accomplishments/certificate/NQJL4LJJEWAX}{Coursera-Certificate}]}
\cvline{January 2020}{\textbf{Improving Deep Neural Networks}[\href{https://www.coursera.org/account/accomplishments/certificate/K2JQBKY8CHZJ}{Coursera-Certificate}]}
\cvline{January 2020}{\textbf{Neural Networks and Deep Learning}[\href{https://www.coursera.org/account/accomplishments/certificate/FRFRSVUFBP9W}{Coursera-Certificate}]}
\cvline{December 2019}{\textbf{Machine Learning}[\href{https://www.coursera.org/account/accomplishments/certificate/VWK8ULNA68UN}{Coursera-Certificate}]}
\cvline{December 2018}{\textbf{Bayesian Statistics}[\href{https://www.coursera.org/account/accomplishments/verify/7HD9KTN289JK}{Coursera-Certificate}]}
\cvline{August 2017}{\textbf{Inferential Statistics}[\href{http://www.coursera.org/account/accomplishments/verify/TF3UNZB67DGH}{Coursera-Certificate}]}


\section{Awards}
\cvline{June 2012}{National Student Conference in Metallic Materials - Awarded best presentation prize for the presentation of PhD project. [\href{https://workspace.imperial.ac.uk/materials/Public/files/Imperial\%20College\%20Department\%20of\%20Materials\%20-\%20Newsletter\%20March\%202013\%20.pdf}{DepartmentLetters.pdf}]}
\cvline{July 2009}{Queen Mary University of London - Awarded the Westfield Trust Prize for outstanding academic achievement,  [\href{http://www.maths.qmul.ac.uk/sites/default/files/3_General_Guidance.pdf}{Awards.pdf}]}
\cvline{May 2006}{QCA Lewisham College - Gym, Exercise and Fitness Knowledge instructor.}
\cvline{July 2005}{Lewisham College - Awarded enrichment certificate in peer mentoring.}

\section{Speaking}
\cvline{June 2014}{Imperial summer seminar series - Talk ``A fundamental problem in computational steels processing''.}
\cvline{December 2013}{International Conference on Processing \& Manufacturing of Advanced Materials - Poster ``A phase-field model for the formation of martensite and bainite'' [\href{http://www.thermec.org/template3s/login/pr-internal/Programme_Book_Final-optimised.pdf}{ThermecProgramme.pdf}]}
\cvline{June 2012}{National Student Conference in Metallic Materials - Talk ``A phase-field model for martensite''.}

\section{Publications \small{(\href{https://imperial.academia.edu/TanselArif/Papers}{\textsc{academia.edu}})}}
\cvline{[1]}{T. T. Arif and R. S. Qin: \textit{A phase-field model for bainitic transformation}, Computational Materials Science \textbf{77} (2013) 230, [\href{http://www.sciencedirect.com/science/article/pii/S0927025613002164}{doi:10.1016/j.commatsci.2013.04.044}].\label{refbain}}
\cvline{[2]}{T. T. Arif and R. S. Qin, \textit{A phase-field Model for the Formation of Martensite and Bainite}, Advanced Materials Research \textbf{922} (2014) 31, [\href{http://www.scientific.net/AMR.922.31}{doi:10.4028/www.scientific.net/AMR.922.31}].}
\cvline{[3]}{Y. Zhao, D. Chen, M. Long, T. Arif and R. Qin, \textit{A three dimensional cellular automata model for dendrite growth with various crystallographic orientations during solidification}, Metallurgical and Materials Transactions B \textbf{45} (2014) 719, [\href{http://link.springer.com/article/10.1007/s11663-013-9960-3/fulltext.html}{doi:10.4028/www.scientific.net/AMR.922.31}].\label{refmeas}}
\end{document}
